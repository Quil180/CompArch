\documentclass{article}

\usepackage{fancyhdr}
\usepackage{extramarks}
\usepackage{amsmath}
\usepackage{amsthm}
\usepackage{amsfonts}
\usepackage{tikz}
\usepackage[plain]{algorithm}
\usepackage{algpseudocode}
\usepackage{listings}
\usepackage{xcolor}
\usepackage{tabularx}
\usepackage{graphicx}
\usepackage{minted}
\usepackage{caption}
\usepackage{subcaption}
\usepackage{booktabs}

% Microtype setup
\let\CheckCommand\providecommand
\usepackage{microtype}
\microtypecontext{spacing=nonfrench}

\newcolumntype{C}{>{\centering\arraybackslash}X}

\usetikzlibrary{automata,positioning,tikzmark}

\definecolor{codegreen}{rgb}{0,0.6,0}
\definecolor{codegray}{rgb}{0.5,0.5,0.5}
\definecolor{codepurple}{rgb}{0.58,0,0.82}
\definecolor{backcolour}{rgb}{0.95,0.95,0.92}

\NewEnvironmentCopy{cminted}{minted}
\AddToHook{env/cminted/begin}{%
  \RecustomVerbatimEnvironment{Verbatim}{BVerbatim}{}%
}

%
% Basic Document Settings
%

\topmargin=-0.45in
\evensidemargin=0in
\oddsidemargin=0in
\textwidth=6.5in
\textheight=9.0in
\headsep=0.25in

\linespread{1.1}

\pagestyle{fancy}
\lhead{Yousef Alaa Awad}
\chead{\hmwkClass\: \hmwkTitle}
\rhead{\firstxmark}
\lfoot{\lastxmark}
\cfoot{\thepage}

\renewcommand\headrulewidth{0.4pt}
\renewcommand\footrulewidth{0.4pt}

%
% Create Problem Sections
%

\setcounter{secnumdepth}{0}
\newcounter{partCounter}
\newcounter{homeworkProblemCounter}
\setcounter{homeworkProblemCounter}{1}

\newcommand{\hmwkTitle}{Homework\ \#5}
\newcommand{\hmwkClass}{Computer Architecture, Section 379}

%
% Title Page
%

\title{
    \vspace{2in}
    \textmd{\textbf{\hmwkClass:\ \hmwkTitle}}\\
    \normalsize\vspace{0.1in}
    \vspace{3in}
}

\author{Yousef Alaa Awad}

\definecolor{hlbrown}{RGB}{255, 220, 180} % Light Orange/Brown for $s1
\definecolor{hlyellow}{RGB}{255, 255, 180} % Light Yellow for group 1 ($t0, etc)
\definecolor{hlgreen}{RGB}{200, 255, 200} % Light Green for group 2 ($t6, etc)
\definecolor{hlteal}{RGB}{200, 240, 255}  % Light Cyan/Teal for group 3 ($t9, etc)

\newcommand{\hl}[2]{\colorbox{#1}{\strut #2}}

\begin{document}

\maketitle
\pagebreak

% Include Picture example
% \includegraphics[width=\textwidth]{q1.png}

\section{1}
\textbf{Given:} \textit{Given the following code:}
\begin{center}
  \begin{minipage}{0.65\linewidth}
    \begin{minted}{gas}
Loop:
  lw   $t0, 0($s1)
  add  $t0, $t0,   $s2
  sub  $t0, $t0,   $s3  # extra dependency
  and  $t4, $t0,   $s4  # added dependent instruction
  or   $t5, $t4,   $s5  # added dependent instruction
  sw   $t5, 0($s1)
  addi $s1, $s1,   -4
  bne  $s1, $zero, Loop
    \end{minted}
  \end{minipage}
\end{center}

\subsection{A) Arrange the loop in the two-issue Slot 1 / Slot 2 table format shown in slide 20, with respect to all hazards. Compute the IPC.}
\begin{table}[H]
    \centering
    \begin{tabular}{@{}lllc@{}}
        \toprule
        \textbf{cycle} & \textbf{} & \textbf{ALU/branch} & \textbf{Load/Store}\\ 
        \midrule
        1 & Loop: & nop                    & lw \$t0, 0(\$s1) \\
        2 &       & addi \$s1, \$s1, -4    & nop              \\
        3 &       & add \$t0, \$t0, \$s2   & nop              \\
        4 &       & sub \$t0, \$t0, \$s3   & nop              \\
        5 &       & and \$t4, \$t0, \$s4   & nop              \\
        6 &       & or \$t5, \$t4, \$s5    & nop              \\
        7 &       & bne \$s1, \$zero, Loop & sw \$t5, 4(\$s1) \\
        \bottomrule
    \end{tabular}
\end{table}
\noindent
\textbf{Performance Calculation:}
\[
\text{IPC} = \frac{8 \text{ instructions}}{7 \text{ cycles}} \approx 1.143\ IpC
\]

\subsection{B) Unroll the loop three times (as opposed to four in slide 23). Show the three copies. Rescheduled loop unrolled using the Slot 1 / Slot 2 table and compute IPC.}
\begin{table}[H]
    \centering
    \begin{tabular}{@{}lllc@{}}
        \toprule
        \textbf{cycle} & \textbf{} & \textbf{ALU/branch} & \textbf{Load/Store} \\ 
        \midrule
        1              & Loop: & nop & lw \$t0, 0(\$s1) \\
        2              & & nop & nop \\
        3              & & add \$t0, \$t0, \$s2 & nop \\
        4              & & sub \$t0, \$t0, \$s3 & nop \\
        5              & & and \$t4, \$t0, \$s4 & nop \\
        6              & & or \$t5, \$t4, \$s5 & lw \$t0, -4(\$s1) \\
        7              & & nop & sw \$t5, 0(\$s1) \\
        8              & & add \$t0, \$t0, \$s2 & nop \\
        9              & & sub \$t0, \$t0, \$s3 & nop \\
        10             & & and \$t4, \$t0, \$s4 & nop \\
        11             & & or \$t5, \$t4, \$s5 & lw \$t0, -8(\$s1) \\
        12             & & nop & sw \$t5, -4(\$s1) \\
        13             & & add \$t0, \$t0, \$s2 & nop \\
        14             & & sub \$t0, \$t0, \$s3 & nop \\
        15             & & and \$t4, \$t0, \$s4 & nop \\
        16             & & or \$t5, \$t4, \$s5 & nop \\
        17             & & addi \$s1, \$s1, -12 & sw \$t5, -8(\$s1) \\
        18             & & bne \$s1, \$zero, Loop & nop \\
        \bottomrule
    \end{tabular}
\end{table}
\vspace{0.5cm}
\noindent
\textbf{Performance Calculation:}
\[
\text{IPC} = \frac{20 \text{ instructions}}{18 \text{ cycles}} \approx 1.111\ IpC
\]

\subsection{C) Apply register renaming following the method shown in slide 26. Use distinct temporaries for each unrolled copy and highlight index changes. Reschedule and compute the IPC.}
\begin{table}[H]
    \centering
    \setlength{\tabcolsep}{8pt} 
    % Changed column structure: Cycle (c) is now first
    \begin{tabular}{@{}clll@{}}
        \toprule
        \textbf{cycle} & \textbf{} & \textbf{ALU/branch} & \textbf{Load/Store} \\ 
        \midrule
        1 & Loop: & addi \hl{hlbrown}{\$s1}, \hl{hlbrown}{\$s1}, -12 & lw \hl{hlyellow}{\$t0}, 0(\$s1) \\
        
        2 &       & nop & lw \hl{hlgreen}{\$t6}, 8(\$s1) \\
              
        3 &       & add \hl{hlyellow}{\$t0}, \hl{hlyellow}{\$t0}, \$s2 & lw \hl{hlteal}{\$t9}, 4(\$s1) \\
              
        4 &       & sub \hl{hlyellow}{\$t0}, \hl{hlyellow}{\$t0}, \$s3 & nop \\
              
        5 &       & and \hl{hlyellow}{\$t4}, \hl{hlyellow}{\$t0}, \$s4 & nop \\
              
        6 &       & or \hl{hlyellow}{\$t5}, \hl{hlyellow}{\$t4}, \$s5 & nop \\
              
        7 &       & add \hl{hlgreen}{\$t6}, \hl{hlgreen}{\$t6}, \$s2 & nop \\
              
        8 &       & sub \hl{hlgreen}{\$t6}, \hl{hlgreen}{\$t6}, \$s3 & nop \\
              
        9 &       & and \hl{hlgreen}{\$t7}, \hl{hlgreen}{\$t6}, \$s4 & nop \\
              
        10 &      & or \hl{hlgreen}{\$t8}, \hl{hlgreen}{\$t7}, \$s5 & nop \\
              
        11 &      & add \hl{hlteal}{\$t9}, \hl{hlteal}{\$t9}, \$s2 & nop \\
              
        12 &      & sub \hl{hlteal}{\$t9}, \hl{hlteal}{\$t9}, \$s3 & nop \\
              
        13 &      & and \hl{hlteal}{\$t10}, \hl{hlteal}{\$t9}, \$s4 & sw \hl{hlyellow}{\$t5}, 12(\$s1) \\
              
        14 &      & or \hl{hlteal}{\$t11}, \hl{hlteal}{\$t10}, \$s5 & sw \hl{hlgreen}{\$t8}, 8(\$s1) \\
              
        15 &      & bne \hl{hlbrown}{\$s1}, \$zero, Loop & sw \hl{hlteal}{\$t11}, 4(\$s1) \\
        \bottomrule
    \end{tabular}
\end{table}
\vspace{0.5cm}
\noindent
\textbf{Performance Calculation:}
\[
\text{IPC} = \frac{20 \text{ instructions}}{15 \text{ cycles}} \approx 1.333\ IpC
\]


\subsection{D) Compute the speedup achieved from part (b) to part (c).}
The performance speedup is as follows:
$$ \frac{IpC}{IpC} = \frac{1.333}{1.111} \approx 1.2\text{ times faster}$$

\pagebreak
\section{2}
\textbf{Given:} \textit{Given the following code:}
\begin{center}
  \begin{minipage}{0.65\linewidth}
    \begin{minted}{gas}
addi $s1, $s0,   16
lw   $t0, 0($s1)
addi $s2, $s0,   200
lw   $t1, 0($s2)
mul  $t2, $t0,   $t1 # hazard chain extension
add  $t3, $t2,   $t4
and  $t5, $t3,   $t6
or   $t7, $t1,   $t5
    \end{minted}
  \end{minipage}
\end{center}

\subsection{A) Identify the hazards and draw the dependency structure in the format used in slide 37.}
The dependencies are as follows...
\begin{table}[H]
    \centering
    \begin{tabular}{@{}ll@{}}
        \toprule
        \textbf{Register} & \textbf{Lines} \\ 
        \midrule
        \$s1 & Lines 1, 2 \\
        \$t0 & Lines 2, 5 \\
        \$s2 & Lines 3, 4 \\
        \$t1 & Lines 4, 5 \\
        \$t1 & Lines 4, 8 \\
        \$t2 & Lines 5, 6 \\
        \$t3 & Lines 6, 7 \\
        \$t5 & Lines 7, 8 \\
        \bottomrule
    \end{tabular}
\end{table}

\subsection{B) Assume both load instructions experience cache misses. Rewrite the code and annotate each line as in slide 38: “cache miss”, “on hold”, or “OK to execute.”}
\begin{table}[H]
    \centering
    % Columns: Line Number (c), Instruction (l), Status (l)
    \begin{tabular}{@{}cll@{}}
        \toprule
        \textbf{Instruction} & \textbf{Status} \\ 
        \midrule
        \texttt{addi\phantom{i} \$s1, \$s0, 16}                & ok \\
        \texttt{lw\phantom{diii} \$t0, 0(\$s1)}      & cache miss \\
        \texttt{addi \$s2, \$s0, 200}               & ok \\
        \texttt{lw\phantom{diii} \$t1, 0(\$s2)}     & cache miss \\
        \texttt{mul\phantom{i} \$t2, \$t0, \$t1}  & on hold \$t1 \\
        \texttt{add\phantom{i} \$t3, \$t2, \$t4}  & on hold \$t2 \\
        \texttt{and\phantom{i} \$t5, \$t3, \$t6}  & on hold \$t3 \\
        \texttt{or\phantom{di} \$t7, \$t1, \$t5} & on hold \$t5 \\
        \bottomrule
    \end{tabular}
\end{table}

\pagebreak
\subsection{C) Before the misses resolve, show the out-of-order execution that can occur, and list the instructions in the reorder buffer with destination, ready/not-ready, and commit status.}
\begin{itemize}
    \item \textbf{Line 1:} Executes successfully.
    \item \textbf{Line 2:} Encountered a cache miss.
    \item \textbf{Line 3:} Executes out-of-order.
    \item \textbf{Line 4:} Encountered a cache miss.
    \item \textbf{Lines 5--8:} Currently on hold/dependent.
\end{itemize}
\begin{table}[H]
    \centering
    \renewcommand{\arraystretch}{1.2} % Slightly more breathing room
    \begin{tabular}{@{}cllll@{}}
        \toprule
        \textbf{\#} & \textbf{Instruction} & \textbf{Destination} & \textbf{Ready/Not Ready} & \textbf{Commit Status} \\ 
        \midrule
        1 & \texttt{addi \$s1, \$s0, 16}                & \$s1 & Ready     & \textbf{Committed} \\
        2 & \texttt{lw\phantom{di}   \$t0, 0(\$s1)}     & \$t0 & Not Ready & Not Committed \\
        3 & \texttt{addi \$s2, \$s0, 200}               & \$s2 & Ready     & Not Committed \\
        4 & \texttt{lw\phantom{di}   \$t1, 0(\$s2)}     & \$t1 & Not Ready & Not Committed \\
        5 & \texttt{mul\phantom{i}   \$t2, \$t0, \$t1}  & \$t2 & Not Ready & Not Committed \\
        6 & \texttt{add\phantom{i}   \$t3, \$t2, \$t4}  & \$t3 & Not Ready & Not Committed \\
        7 & \texttt{and\phantom{i}   \$t5, \$t3, \$t6}  & \$t5 & Not Ready & Not Committed \\
        8 & \texttt{or\phantom{di}    \$t7, \$t1, \$t5} & \$t7 & Not Ready & Not Committed \\
        \bottomrule
    \end{tabular}
    \caption*{Status of instructions in the Reorder Buffer / Pipeline}
\end{table}

\end{document}
